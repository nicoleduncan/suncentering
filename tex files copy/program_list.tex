\RequirePackage[l2tabu, orthodox]{nag}
\documentclass[10pt]{article}
\usepackage[T1]{fontenc}
\usepackage{amsmath,amsfonts,amssymb}
\usepackage{mathtools}
\usepackage{color,soul}
\usepackage{fullpage}
\usepackage{enumerate}
\usepackage{graphicx}
\usepackage[colorlinks=true,urlcolor=blue]{hyperref}
\usepackage{subcaption}
\usepackage[activate={true,nocompatibility},final,tracking=true,kerning=true,spacing=true,factor=1100,stretch=10,shrink=10]{microtype}

\definecolor{Light}{gray}{.90}
\sethlcolor{Light}

\title{Rough List of Programs}
\author{Jeren Suzuki}
\date{Last Edited \today}

\begin{document}

\maketitle
\pagenumbering{Roman}
\tableofcontents
\addcontentsline{toc}{section}{Introduction}
\clearpage
\pagenumbering{arabic}

\section*{Introduction} % (fold)
\label{sec:introduction}
I always get confused which programs are which so I'm just putting them all here
% section introduction (end)

\section{List of Programs} % (fold)
\label{sec:list_of_programs}

\begin{enumerate}
	\item \hl{\texttt{barkbark}}\\
		makes some plots comparing the attempts of finding a threshold for our image
	\item \hl{\texttt{bordercheck}}\\
		Makes a binary mask 1 pixel thick on the border, checks if there are any pixels that are above the mode of the image. If so, returns a 0. If okay, 1.
	\item \hl{\texttt{circscancrop}}\\
		Using the position of the center sun, scans in a circle at a specified radius until it picks up the auxiliarry suns and crops appropriately.
	\item \hl{\texttt{cropme}}\\
		Crops a certain pixel legth off the border
	\item \hl{\texttt{edgefidcheck}}\\
		Incomplete: Checks the bordering 6 pixels of an image for fiducials. If it finds one, it does something.
	\item \hl{\texttt{fastcenter}}\\
		Uses the sorted image to return very rough center positions of the suns. Fast.
	\item \hl{\texttt{galapagos}}\\
		Earlier version of barkbark I think
	\item \hl{\texttt{getstruct}}\\
		Makes 3 structures, 1 for each sun. Each structure holds the center position as well as the threshold used to find it
	\item \hl{\texttt{histosmoothed}}\\
		Tries looking at the histogram of the 2D image instead of the sorted array, doesn't work well.
	\item \hl{\texttt{kahuna}}\\
		Calls getstruct and is more or less a calling program that loads the parameter table
	\item \hl{\texttt{last6pixels}}\\
		Makes some plots of what the bordering 6 pixels of a fiducial look like as you crop it off the edge of the iamge
	\item \hl{\texttt{limbfit}}\\
		Finds the center of a sun using chords
	\item \hl{\texttt{makelimbstrips}}\\
		Makes 5 limb-only arrays fron the 5 chord arrays
	\item \hl{\texttt{makestrips}}\\
		Makes 5 chord arrays centered on the specific sun
	\item \hl{\texttt{morescratch}}\\
		Messing with convol() and stuff to find the threshold, obsolete
	\item \hl{\texttt{quickfidmask}}\\
		Used to find a quick CoM of a shape below a threshold
	\item \hl{\texttt{quickmask}}\\
		Used to find a quick CoM of a shape above a threshold
	\item \hl{\texttt{scratch}}\\
		Another thing to find the threshold
	\item \hl{\texttt{setpeak}}\\
		Find the boundaries of the humps in the sorted image 
	\item \hl{\texttt{setthresh}}\\
		Finds the thresh of the 1D array at the hump intersection point. Very barebones, actually.
	\item \hl{\texttt{whichcropmethod}}\\
		Decides whether or not to use quickmask or circscancrop when finding the center of the suns
\end{enumerate}


% section list_of_programs (end)

\end{document}










