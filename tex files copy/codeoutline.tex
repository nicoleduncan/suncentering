\RequirePackage[l2tabu, orthodox]{nag}
\documentclass[10pt]{scrartcl}
% \documentclass[10pt]{article}
\usepackage[T1]{fontenc}
\usepackage{amsmath,amsfonts,amssymb}
\usepackage{mathtools}
\usepackage{color,soul}
\usepackage{enumerate}
\usepackage[margin=2cm]{geometry}
\usepackage{graphicx}
\usepackage[colorlinks=true,urlcolor=blue]{hyperref}
\usepackage{floatrow}
\usepackage{deluxetable}
\usepackage{verbatim}
\usepackage{fancyvrb}
\usepackage{listings}
\usepackage{calc}
\usepackage[font=small]{caption}
\usepackage[font=scriptsize]{subcaption}
\usepackage[activate={true,nocompatibility},final,tracking=true,kerning=true,spacing=true,factor=1100,stretch=10,shrink=10]{microtype}
\SetTracking{encoding={*}, shape=sc}{40}

\floatsetup{ 
  heightadjust=object,
  valign=t
}

\definecolor{Light}{gray}{.90}
\sethlcolor{Light}

\title{Code Outline}
\author{Jeren Suzuki}
\date{Last Edited \today}

\begin{document}

\maketitle
\pagenumbering{Roman}
\tableofcontents
\clearpage
\pagenumbering{arabic}

\section{Introdution} % (fold)
\label{sec:introdution}
This document describes roughly how the program finds the centroids of a multi-sun image.\\

The program is broken into X parts:

\begin{enumerate}
    \item Finding Centers
    \item Eliminate Near-Edge Suns
    \item Find Fiducials
\end{enumerate}
% section introdution (end)

\section{Finding Centers} % (fold)
\label{sec:finding_centers}
    The main bulk of the program is to find the centroids of the suns in the image. First we start off find the thresholds of the image which incidentally takes the longest amount of time. Once we have the thresholds, we create a binary mask of the image above a threshold. 
% section finding_centers (end)
\\


\end{document}










