\documentclass[10pt]{scrartcl}
% \documentclass[10pt]{article}
\usepackage[T1]{fontenc}
\usepackage{amsmath,amsfonts,amssymb}
\usepackage{mathtools}
\usepackage{color,soul}
\usepackage{fullpage}
\usepackage{enumerate}
\usepackage{graphicx}
\usepackage[colorlinks=true,urlcolor=blue]{hyperref}
% \usepackage[justification=center]{caption}
\usepackage{subcaption}
\usepackage{deluxetable}
\usepackage{verbatim}
\usepackage{fancyvrb}
\usepackage{listings}


\definecolor{Light}{gray}{.90}
\sethlcolor{Light}


\lstset{%
language=IDL,                   % choose the language of the code
basicstyle=\footnotesize\sffamily,%\ttfamily\footnotesize,       % the size of the fonts that are used for the code
numbers=left,                   % where to put the line-numbers
numberstyle=\footnotesize,      % the size of the fonts that are used for the line-numbers
stepnumber=1,                   % the step between two line-numbers. If it is 1 each line will be numbered
numbersep=5pt,                  % how far the line-numbers are from the code
showspaces=false,               % show spaces adding particular underscores
showstringspaces=false,         % underline spaces within strings
showtabs=false,                 % show tabs within strings adding particular underscores
% frame=single,                   % adds a frame around the code
backgroundcolor=\color{Light},
columns=flexible,
tabsize=2,                      % sets default tabsize to 2 spaces
captionpos=b,                   % sets the caption-position to bottom
breaklines=true,                % sets automatic line breaking
breakatwhitespace=false,        % sets if automatic breaks should only happen at whitespace
escapeinside={\%*}{*)}          % if you want to add a comment within your code
}

\title{Final Documentation Draft}
\author{Jeren Suzuki}
\date{Last Edited \today}

\begin{document}

\maketitle
\pagenumbering{Roman}
\tableofcontents
\clearpage
\pagenumbering{arabic}

\section{Introduction} % (fold)
\label{sec:introduction}
Starting with an image of at most three suns on a fiducial grid, we find the centers of the suns and their relative position to the fiducials (which provide a physical distance calibration). The program deems suns too close to the edge or suns partially cut off as unfit for centering. 
% section introduction (end)

\section{Introduction cont.} % (fold)
\label{sec:intro_cont}
% This outline of steps aims to lay out the basic form we Not sure how in-depth to make the list so I'll just mark it for later.

%%%%%%%%%%%%%%%%%%%%%%%%%%%%%%%%%%%%%%%%%%%%%%%%%%%%%%%%%%%%%%%%%%
%
%
%
%
%
%%%%%%%%%%%%%%%%%%%%%%%%%%%%%%%%%%%%%%%%%%%%%%%%%%%%%%%%%%%%%%%%%%

\subsection{Program List} % (fold)
\label{sub:program_list}

\begin{enumerate}
    \item \hl{\texttt{defparams}}\\
        Reads in parameter table
    \item \hl{\texttt{defsysvarthresh}}\\
        Sets thresholds
    \item \hl{\texttt{idsuns}}\\
        Identifies shapes in image. For each shape, calculate average value/max and assign region number depending on how bright it is
    \item \hl{\texttt{everysun}}\\
        For each shape, find center 
    \item \hl{\texttt{quickmask}}\\
        Using a mask of pixel values above a threshold, find center of mass of mask
    \item \hl{\texttt{picksun}}\\
        Determine if center is too close to edge of image; if so, mark as a partial sun and refrain from further analysis
    \item \hl{\texttt{centroidwholesuns}}\\
        Depending on which suns are partial, only limb-fit the whole suns
    \item \hl{\texttt{fourpixfit}}\\
        Fit line to a four pixel limb-profile centered around where it crosses a threshold
    \item \hl{\texttt{makeslimlimbstrips}}\\
        Make limb strips 4 pixels long 
    \item \hl{\texttt{makestrips}}\\
        Make strips centered around solar center
    \item \hl{\texttt{para\_fid}}\\
        Find fiducials in each limb-fitted sun, also use parabolic peak-fitting to calculate sub-pixel positions of fiducials
\end{enumerate}

Has to be a cleaner way of doing this
% subsection program_list (end)

\begin{enumerate}
    \item Load Image
    \item Read parameters from pblock.txt
    \item Sort image and cut off top .1\% of pixels (top 1\% was actually too much)
    \item Smooth, take deriv, smooth again, take deriv again of sorted array, find peaks that correspond to difference solar regions and their thresholds
    \item Mask image above thresholds to find centers of every shape, regardless of partial or not
    \item If center of shape is within a certain distance to edge of image, mark as partial and cease further analysis
    \item Crop remaining whole suns
    \item Extract 5 strips centered around cropped solar center for both X and Y direction
    \item Extract a pair of limb strips for each long strip
    \item Applt linear fit to limb profile
    \item Mark position where fit crosses threshold 
    \item Use new threshold-crossing position to calculate chord lengths
    \item Average midpoints of chords to find limb-fitted centers
    % \item Recrop image and calculate row/column sums 
    % \item Subtract smoothed sums from original sum array to return fiducial row/column positions
    \item Analyze the cropped image for fiducials
    \item Using the fiducial positions, we compare the solar positions we calculated to a position defined by the physical setup.
\end{enumerate}

\begin{deluxetable}{cllllllll}
    \tablecaption{Final data structure of solar region}
    \tablecolumns{4}
    \tabletypesize{\scriptsize}
    \tablewidth{0pt}
    \tablehead{ 
      \colhead{Name} %
    & \colhead{Type} %
    & \colhead{Value} %
    & \colhead{Notes}
    }
    \startdata
    \hline
    XPOS
    & FLOAT
    & 210.522
    & Rough calculation using a simple masking method\\
    %
    YPOS
    & FLOAT
    & 166.702
    & ''\\
    %
    REG
    & INT
    & 1
    & Region ID \#: 1 is 100\%, 2 is 50\%, 3 is 25\%\\
    %
    THRESH
    & FLOAT
    & 106.000
    & Threshold calculated from sorting array and taking derivatives.\\ & & & Used in both finding rough X-Y center as well as the\\ & & & threshold for limb-fitting.\\
    %
    PARTIAL
    & FLOAT
    & 0.
    & Flag that determines if the solar region is cut off on the edge or not.\\ & & & 0 means that it is not cut off \\
    %
    XSTRIPS
    & STRUCTURE
    & -> WHOLEXSTRIPS Array[5]
    & Strucutre containing the strips of whole solar data\\ & & & bound by a cropped region chosen by XPOS and YPOS\\
    %
    YSTRIPS
    & STRUCTURE
    & -> WHOLEYSTRIPS Array[5]
    & ''\\
    %
    LIMBXSTRIPS
    & STRUCTURE
    & -> LIMBXSTRIPS Array[5]
    & LIMBSTRIPS contains a pair of arrays, ENDPOINTS and \\ & & & STARTPOINTS that mark the limbs of each strip of data from \\ & & & X/YSTRIPS\\
    %
    LIMBYSTRIPS
    & STRUCTURE
    & -> LIMBYSTRIPS Array[5]
    & ''\\
    %
    LIMBXPOS
    & FLOAT
    & 210.710
    & Center calculated from LIMBXSTRIPS\\
    %
    LIMBYPOS
    & FLOAT
    & 167.172
    & ''\\
    %
    NPIX
    & FLOAT
    & 26680.0
    & Number of pixels above threshold
    \enddata
\label{structtable}
\end{deluxetable}

This is the form of the fiducial structure containing the positions and xub-pixel positions of fiducials for each solar region.
\begin{lstlisting}
>> help,*(bbb[0])
** Structure <260a348>, 2 tags, length=180, data length=178, refs=1:
   REG             INT              1
   FIDARR          STRUCT    -> FIDPOS Array[11]
>> help,(*(bbb[0])).fidarr,/str
** Structure FIDPOS, 4 tags, length=16, data length=16:
   X               FLOAT           50.0000
   Y               FLOAT           132.000
   SUBX            FLOAT           50.8438
   SUBY            FLOAT           133.291
\end{lstlisting}

% section intro_cont (end)

\section{Setting Up Parameters} % (fold)
\label{sec:setting_up_parameters}
Before we analyze the solar image, we load a parameter table and assign values. 
\begin{lstlisting}
scan_width 10               ; Distance to next chord when picking chords to limb-fit
sundiam 70                  ; Approx Solar diameter, deprecated
nstrips 5                   ; Number of pairs of solar chords to limb-fit per direction
ministrip_length 4          ; Length of limb profile to linear fit
crop_box 120                ; Half-width of box used to find fiducials in
elim_perc 1                 ; Percentage of highest pixels to eliminate when finding threshold
n_smooth 900                ; Elements to smooth by when finding threshold 
soldiskr 60                 ; Deprecated
border_pad 50               ; If solar center is within this value of border, marked as a partial sun
triangle_size .25           ; Percentage of image height to use for triangle sides for making clipped-bottom-corner mask
fid_smooth_thresh -150      ; Threshold to determine row/column positions of fiducials
onedsumthresh 80            ; Once looking at fiducial candidates, look at 1D sum of smaller fiducial crop and threshold difference of smoothed array - original array by this
disk_brightness 15          ; Arbitrary pixel brightness to eliminate bright fiducial candidates which are on the solar disk but are not on a fiducial
fid_crop_box 15             ; Half-width of box used to analyze fiducials
fid_smooth_candidates 15    ; Smoothing paramater for 1D sums of fiducial candidates 
\end{lstlisting}

% section setting_up_parameters (end)

\end{document}










