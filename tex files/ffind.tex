\documentclass[10pt]{scrartcl}
% \documentclass[10pt]{article}
\usepackage[T1]{fontenc}
\usepackage{amsmath,amsfonts,amssymb}
\usepackage{mathtools}
\usepackage{color,soul}
\usepackage{fullpage}
\usepackage{enumerate}
\usepackage{graphicx}
\usepackage[colorlinks=true,urlcolor=blue]{hyperref}
\usepackage{caption}
\usepackage{subcaption}
\usepackage{floatrow}
\usepackage{deluxetable}
\usepackage{verbatim}
\usepackage{fancyvrb}
\usepackage{listings}
\usepackage{calc}

\floatsetup{ 
  heightadjust=object,
  valign=t
}

\definecolor{Light}{gray}{.90}
\sethlcolor{Light}

\lstset{%
language=IDL,                   % choose the language of the code
basicstyle=\footnotesize\sffamily,%\ttfamily\footnotesize,       % the size of the fonts that are used for the code
numbers=left,                   % where to put the line-numbers
numberstyle=\footnotesize,      % the size of the fonts that are used for the line-numbers
stepnumber=1,                   % the step between two line-numbers. If it is 1 each line will be numbered
numbersep=5pt,                  % how far the line-numbers are from the code
showspaces=false,               % show spaces adding particular underscores
showstringspaces=false,         % underline spaces within strings
showtabs=false,                 % show tabs within strings adding particular underscores
% frame=single,                   % adds a frame around the code
backgroundcolor=\color{Light},
columns=flexible,
tabsize=2,                      % sets default tabsize to 2 spaces
captionpos=b,                   % sets the caption-position to bottom
breaklines=true,                % sets automatic line breaking
breakatwhitespace=false,        % sets if automatic breaks should only happen at whitespace
escapeinside={\%*}{*)}          % if you want to add a comment within your code
}

\title{Fiducial matching method}
\author{Jeren Suzuki}
\date{Last Edited \today}

\begin{document}

\maketitle
\pagenumbering{Roman}
% \tableofcontents
\clearpage
\pagenumbering{arabic}

The best way I came up with to find the fiducials is to make an NxN matrix with the rows/columns as N fiducial positions. For example, we have 4 fiducials that we call A, B, C, and D. The matrix looks like:

\[ \begin{array}{ccccc}
 & A & B & C & D \\
A & & & &  \\
B & & & &  \\
C & & & &  \\
D & & & &  \end{array} \] 

We then fill it in to be
\[ \begin{array}{cccc}
AA & AB & AC & AD \\
BA & BB & BC & BD \\
CA & CB & CC & CD \\
DA & DB & DC & DD \end{array}\]

These are the combinations of all possible chord lengths with the endpoints being the fiducial indexes. Now we rule out AA, BB, CC, and DD since the distance from A to A is 0, etc. We eliminate duplicate chords to get:

\[ \begin{array}{cccc}
0 & AB & AC & AD \\
0 & 0 & BC & BD \\
0 & 0 & 0 & CD \\
0 & 0 & 0 & 0 \end{array}\]

To get the actual chord lengths, we use the formula $D = \sqrt{x^2 + y^2}$ with:

\[ x = \begin{array}{cccc}
0 & \overline{A_xB_x} & \overline{A_xC_x} & \overline{A_xD_x} \\
0 & 0 & \overline{B_xC_x} & \overline{B_xD_x} \\
0 & 0 & 0 & \overline{C_xD_x} \\
0 & 0 & 0 & 0 \end{array}\]
\[ y = \begin{array}{cccc}
0 & \overline{A_yB_y} & \overline{A_yC_y} & \overline{A_yD_y} \\
0 & 0 & \overline{B_yC_y} & \overline{B_yD_y} \\
0 & 0 & 0 & \overline{C_yD_y} \\
0 & 0 & 0 & 0 \end{array}\]



Here is where I'm getting myself a little confused, do we have a list of chord lengths between fiducials or a list of fiducial positions? My thinking is that if they are chord lengths, then we can just match up the distances between fiducials but since each chord is a pair of fiducials, we don't know for a single chord which fiducial is on which end. We'd have to iterate through a bunch of fiducial chords to figure out which fiducial we ID is the fiducial in the table. \\

If we're working with a list of fiducial positions then I can just look up the distance of the fiducial from the origin. Any fiducial within, say, 1\% of a fiducial position on our table is considered a match.\\

I don't see any clear benefits/cons between the two types of lists; is there one?

\end{document}
