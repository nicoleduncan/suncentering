\RequirePackage[l2tabu, orthodox]{nag}

\documentclass[10pt]{scrartcl}
\usepackage[T1]{fontenc}
\usepackage{amsmath,amsfonts,amssymb}
\usepackage{mathtools}
\usepackage{cancel}
\usepackage{color,soul}
\usepackage[margin=2cm,letterpaper]{geometry}
\usepackage{enumerate}
\usepackage{graphicx}
\usepackage[colorlinks=true,urlcolor=blue]{hyperref}
\usepackage{floatrow}
\usepackage{deluxetable}
\usepackage{verbatim}
\usepackage{fancyvrb}
\usepackage{listings}
\usepackage{calc}
\usepackage{xfrac}
\usepackage{cleveref}
\usepackage[sharp]{easylist}
\usepackage[font=small]{caption}
\usepackage[font=scriptsize]{subcaption}
\usepackage[activate={true,nocompatibility},final,tracking=true,kerning=true,spacing=true,factor=1100,stretch=10,shrink=10]{microtype}
% \usepackage[tocindentauto]{tocstyle}
\SetTracking{encoding={*}, shape=sc}{40}

\microtypecontext{spacing=nonfrench}

\floatsetup{ 
  heightadjust=object,
  valign=t
}

\definecolor{Light}{gray}{.90}
\sethlcolor{Light}

\lstset{%
language=IDL,                       % choose the language of the code
basicstyle=\footnotesize\sffamily,  % the size of the fonts that are used for the code
numbers=left,                       % where to put the line-numbers
numberstyle=\footnotesize,          % the size of the fonts that are used for the line-numbers
stepnumber=1,                       % the step between two line-numbers. If it is 1 each line will be numbered
numbersep=5pt,                      % how far the line-numbers are from the code
showspaces=false,                   % show spaces adding particular underscores
showstringspaces=false,             % underline spaces within strings
showtabs=false,                     % show tabs within strings adding particular underscores
frame=single,                       % adds a frame around the code
backgroundcolor=\color{Light},
columns=flexible,
tabsize=2,                          % sets default tabsize to 2 spaces
captionpos=b,                       % sets the caption-position to bottom
breaklines=true,                    % sets automatic line breaking
breakatwhitespace=false,            % sets if automatic breaks should only happen at whitespace
escapeinside={\%*}{*)}              % if you want to add a comment within your code
}

\title{What Calls What}
\author{Jeren Suzuki}
\date{Last Edited \today}

\begin{document}

\pagenumbering{gobble}
\maketitle
\tableofcontents
\addcontentsline{toc}{section}{Introduction}
\clearpage
\pagenumbering{arabic}

\section*{Introduction}
\label{sec:intro}
Code-calling structure. Once written out, I realized that I have repeated a few steps. As of \today, the code works, but perhaps removing the second call to idfids and organizing cyoalimbstrips will make the code a bit faster.


\section{Code Tree} % (fold)
\label{sec:code_tree}

The main program called is alpha, so I won't include that here. I will only be listing user defined functions because listing all the IDL functions would be silly.\\

\begin{easylist}
# \textbf{defparams} - Defines system parameters from an input parameter .txt file
# \textbf{defsysvarthresh} - Defines solar thresholds
## \textbf{idsuns} - Defines which pixels correspond to which solar regions, does not use threshold values
## \textbf{setbetterpeak} - Finds peaks in 2nd derivative of sorted 2d image to get thresholds
# \textbf{everysun} - Using thresholds, finds centers of every sun-shaped entity, regardless of if it is a partial or too close to the edge
## \textbf{idsuns} - In this scenario, needed to figure out how many suns to center (since the only input is the starting image)
# \textbf{picksun\_rot} - Makes sure the centers found aren't too close to the edges or the bottom two corners
# \textbf{centroidwholesuns} - A secondary ``wrapping'' program that organizes three important functions in one area; more of an organizational program
## \textbf{para\_fid} - Finds all fiducials for a sun and then uses a parabolic fit to get a better position
## \textbf{npixfit} - Uses a linear fit to find where the fitted solar limb crosses a threshold. A chord is drawn in a row/column of sun data that is bounded by two limb-fitted crossing points of the threshold. The center is calculated from an average of chord midpoints.
### \textbf{cyoalimbstrips} - Feeds clean limb structure information into npixfit. Technically we have already found limb data so we can probably take this part out.
#### \textbf{makestrips} - Makes strips of solar data from a cropped image of a sun. The strips extend the entire length/width of the cropped image. There strips of solar data are centered around the masked center of the sun.
##### \textbf{micro\_makelimbstrips} - Calculates limb positions based on threshold crossing points
##### \textbf{copy\_limb\_struct} - copies limb structure information into a larger structure
## \textbf{best4} - Takes the list of fiducials and narrows them down to the 4 closest to disk-center
# \textbf{idfids} - Identifies the best 4 fiducials according to a catalog of distances between fiducials

\end{easylist}
% section code_tree (end)


\end{document}